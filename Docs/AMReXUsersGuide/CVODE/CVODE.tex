\amrex\ supports ODE integration using the \cvode\ solver,\footnote{\url{https://computation.llnl.gov/projects/sundials/cvode}} which is part of the \sundials\ framework.\footnote{\url{https://computation.llnl.gov/projects/sundials}}
\cvode\ contains solvers for stiff and non-stiff ODEs, and as such is well suited for solving e.g., the complex chemistry networks in combustion simulations, or the nuclear reaction networks in astrophysical simulations.

Most of \cvode\ is written in C, but many functions also come with two distinct Fortran interfaces.
One interface is \fcvode, which is bundled with the stable release of \cvode.
Its usage is described in the CVODE documentation (\url{https://computation.llnl.gov/sites/default/files/public/cv_guide.pdf}).

The second, newer Fortran interface to \cvode\ uses the \texttt{iso\_c\_binding} feature of the Fortran 2003 standard to implement a ``thinner,'' more direct interface to the C functions in \cvode.
In contrast to the \fcvode\ interface, the Fortran 2003 interface calls C functions directly, with identical arguments.
When compiling \cvode, one need not build the Fortran interface to \cvode\ at all to use this new interface.
The examples provided in \amrex\ use this new interface.

To use \cvode\ in an \amrex\ application, follow these steps:

\begin{enumerate}

  \item Obtain the \cvode\ source code, which is hosted here: \url{https://computation.llnl.gov/projects/sundials/sundials-software}.

  One can download either the complete \sundials\ package, or just the \cvode\ components.

  \item Unpack the \cvode/\sundials\ tarball, and create a new ``build'' directory (it can be anywhere).

  \item Navigate to the new, empty build directory, and type

  \begin{verbatim}
  cmake \
    -DCMAKE_INSTALL_PREFIX:PATH=/path/to/install/dir \
    /path/to/cvode/or/sundials/top/level/source/dir
  \end{verbatim}

  The \texttt{CMAKE\_INSTALL\_DIR} option tells CMake where to install the libraries.
  Note that CMake will attempt to deduce the compilers automatically, but respects certain environment variables if they are defined, such as \texttt{CC} (for the C compiler), \texttt{CXX} (for the C++ compiler), and \texttt{FC} (for the Fortran compiler).
  So one may modify the above CMake invocation to be something like the following:

  \begin{verbatim}
  CC=/path/to/gcc \
  CXX=/path/to/g++ \
  FC=/path/to/gfortran \
    cmake \
    -DCMAKE_INSTALL_PREFIX:PATH=/path/to/install/dir \
    /path/to/cvode/or/sundials/top/level/source/dir
  \end{verbatim}

  \item In the \texttt{GNUmakefile} for the \amrex\ application which uses the Fortran 2003 interface to \cvode, add \texttt{USE\_CVODE = TRUE}, which will compile the Fortran 2003 interfaces and link the \cvode\ libraries.
  Note that one must define the \texttt{CVODE\_LIB\_DIR} environment variable to point to the location where the libraries are installed.
\end{enumerate}
